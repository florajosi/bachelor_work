\documentclass[a4paper,11pt,bibtotocnumbered]{article}
\usepackage[utf8]{inputenc}
\usepackage[T1]{fontenc}
\usepackage[english]{babel}

\linespread{1.5} %Zeilenabstand

\usepackage[a4paper, left=3cm, right=3cm, top=2.5cm]{geometry} %Seitenränder

\usepackage{booktabs}
\usepackage{tabularx}
\usepackage{float}
\usepackage{longtable, tabu}
\usepackage{multirow}
\usepackage{rotating} %rotation von Zellen
\newcommand\tabrotate[1]{\begin{turn}{90}\rlap{#1}\end{turn}} %rotation von Zellen


\usepackage{tocbibind} %Literaturverz. in Inhaltsverz.


\usepackage{romannum} %römische Zahlen im Fließtext

\usepackage{titling}
\newcommand{\subtitle}[1]{%
  \posttitle{%                            Untertitel
    \par\end{center}
    \begin{center}\large#1\end{center}
    \vskip0.5em}%
}

\usepackage{graphicx}

\setlength{\parindent}{0pt}%kein Einrücken

\usepackage{hyperref}

\setcounter{tocdepth}{5}
\setcounter{secnumdepth}{5}%Gliedrungstiefe im Inhaltsverzeichnis

\usepackage[skip=3pt,font={small}]{caption}

\bibliographystyle{unsrt}

\captionsetup{format=hang, labelfont=bf, textfont=scriptsize} %caption fett

\begin{document}


\title{Biochemistry of Astrogliosis Models \textit{in vitro}}

\author{Flora Joswphine Jung\\
403500}
\date{\today}


\begin{titlepage}
	\centering
	\begin{figure}
   \begin{minipage}[b]{.15\linewidth} % [b] => Ausrichtung an \caption
      \includegraphics[width=\linewidth]{Logo.png}
   \end{minipage}
   \hspace{.7\linewidth}% Abstand zwischen Bilder
   \begin{minipage}[b]{.2\linewidth} % [b] => Ausrichtung an \caption
      \includegraphics[width=\linewidth]{LogoCharite.png}
   \end{minipage}
\end{figure}
	%\includegraphics[width=0.2\textwidth]{Logo.png}\par\vspace{1cm}
	{\scshape\LARGE Master Thesis\par}
	\vspace{1cm}
	{\scshape\Large Flora Josephine Jung \par}
	\vspace{1.5cm}
	{\huge\bfseries Biochemistry of Astrogliosis Models \textit{in vitro}  \par}
	\vspace{2cm}
	{\Large\itshape SoSe 2020 \par}
	\vspace{2cm}
	{\Large Technische Universität Berlin \par}
	\vspace{.5cm}
	{\Large\itshape Institut für Chemie \par}
	\vspace{2cm}
	{\Large Charité Universitätsmedizin Berlin\par}
	\vspace{.5cm}
	{\Large\itshape Institut für Biochemie \par}
	\vfill
	\date
	\vfill
\end{titlepage}

\pagestyle{plain}
\newpage
\pagenumbering{Roman}

\begingroup
\renewcommand*{\thesection}{\Roman{section}}

\newpage
\section{Discussion}

Astrogliosis is correlated with several neuropathological conditions like trauma, ischemic damage and neuroinflammation or neurodegenerative diseases such as \textsc{Alzheimer}s disease, \textsc{Parkinson}s disease and multiple sclerosis~\cite{Pekny2014}. So understanding the molecular processes during astrogliosis is crucially important to develop a holistic understanding of these pathologies. 

\subsection{Cell Culture of stellate Astrocytes}

To do so we established a new cell culture model based on the so called AWESAM-protocoll~\cite{Wolfes2018} in our lab to stablely produce \textit{in vivo}-like, stellate astrocytes. 

We could validate that our astrocyte cultures were free from neurons and microglia via immunostainings and immunoblotting experiments. We could observe a rather high abundance of GFAP aswell as of GST1 and Sox9, which are all well known Astrocyte markers~\cite{Schiweck2018}.

%which is a marker of reactive astrocytes, which indicates that albeit their stellate morphology and \textit{in vivo}-likeness these astrocytes have an \textit{a priori} reactivity.

Map2, a neuron dendrite marker, was almost not detectable indicating a pure astrocytic culture.

Microglia could be selectively removed by performing a LME treatment. LME is internalized by microglia and induces lysosomal disruption and subsequent apoptosis~\cite{Jebelli2015}. The absence of microglia could be validated by immunoblotting experiments.


\newpage
\renewcommand\bibname{}
\bibliography{josephine_bachelor_work.bib}


\end{document}
